\documentclass[leter]{amsart}
\usepackage[pdftex]{geometry}
\usepackage[utf8]{inputenc}

\usepackage{listings}
\usepackage{amssymb}
\usepackage{amsmath}
\usepackage{multicol}
\usepackage{courier}
\usepackage{fancyhdr}

\lstset{language=C++}
\lstset{columns=fullflexible}
\lstset{basicstyle=\scriptsize\ttfamily}
\lstset{showspaces=false}

\geometry{landscape}
\setlength{\topmargin}{-0.25in}
\setlength{\oddsidemargin}{0in}
\setlength{\evensidemargin}{0in}
\setlength{\columnsep}{1in}
\setlength{\columnseprule}{0.2pt}
\textwidth 9.5in

\pagestyle{fancy}
\lhead{Universidad de La Habana - UH}
\chead{}
\rhead{\thepage}
\lfoot{}
\cfoot{}
\rfoot{}

\newcommand{\includecpp}[2]{
  \begin{multicols}{2}
    [\subsection{#1}\ ]
    \begin{lstlisting}[language=C++]
        #2
    \end{lstlisting}
  \end{multicols}
}

\begin{document}

\section{DataStructure}
Abi
\begin{multicols}{2}
\begin{lstlisting}[language=C++]
class Abi
{

private:
    vi p;
    int _size;

    int ls_one(int i) { return i & (-i); }

public:
    Abi(int n)
    {
        _size = n;
        p.assign(n + 1, 0);
    }

    int rsq(int k)
    {
        int sum = 0;

        for (int i = k; i > 0; i -= ls_one(i))
        {
            sum += p[i];
        }

        return sum;
    }

    int sum(int a, int b) { return rsq(b) - rsq(a - 1); }

    void adjust_sum(int k, int v)
    {
        for (int i = k; i < p.size(); i += ls_one(i))
            p[i] += v;
    }

    int size()
    {
        return _size;
    }
};

\end{lstlisting}
\end{multicols}
Segment Tree
\begin{multicols}{2}
\begin{lstlisting}[language=C++]
class SegmentTree
{
private:
    vi values;

    vi p_values;
    int n;

    int left(int p) { return p << 1; };

    int right(int p) { return (p << 1) + 1; }

    int simple_node(int index) { return values[index]; }

    int prop(int x, int y) { return x + y; }

    void build(int p, int l, int r)
    {
        if (l == r)
        {
            p_values[p] = simple_node(l);
            return;
        }

        build(left(p), l, (l + r) / 2);
        build(right(p), (l + r) / 2 + 1, r);

        p_values[p] = prop(p_values[left(p)], p_values[right(p)]);
    }

    void set(int p, int l, int r, int i, int v)
    {
        if (l == r)
        {
            values[l] = v;
            p_values[p] = simple_node(l);
            return;
        }

        if (i <= (l + r) / 2)
            set(left(p), l, (l + r) / 2, i, v);
        else
            set(right(p), (l + r) / 2 + 1, r, i, v);

        p_values[p] = prop(p_values[left(p)], p_values[right(p)]);
    }

    int query(int p, int l, int r, int lq, int rq)
    {
        if (lq <= l && r <= rq)
            return p_values[p];

        int l1 = l, r1 = (l + r) / 2;
        int l2 = (l + r) / 2 + 1, r2 = r;

        if (l1 > rq || lq > r1)
            return query(right(p), l2, r2, lq, rq);
        if (l2 > rq || lq > r2)
            return query(left(p), l1, r1, lq, rq);

        int lt = query(left(p), l1, r1, lq, rq);
        int rt = query(right(p), l2, r2, lq, rq);

        return prop(lt, rt);
    }

public:
    SegmentTree(vi &a)
    {
        values = a;
        n = a.size();
        p_values.assign(4 * n, 0);
        build(1, 0, n - 1);
    }

    int query(int i, int j) { return query(1, 0, n - 1, i, j); }

    void set(int i, int v) { set(1, 0, n - 1, i, v); }

    int get(int i) { return values[i]; }
};

class SegmentTreeLazy
{
private:
    vi values;
    vector<bool> lazy;
    vi l_values;
    vi p_values;
    int n;

    int left(int p) { return p << 1; };

    int right(int p) { return (p << 1) + 1; }

    int simple_node(int index) { return values[index]; }

    int prop(int x, int y) { return x + y; }

    int prop_lazy(int x, int y) { return x + y; }

    int prop_lazy_up(int x, int y, int s) { return x + y * s; }

    void update_lazy(int p, int l, int r)
    {
        if (l == r)
        {
            values[l] = prop_lazy(values[l], l_values[p]);
        }

        p_values[p] = prop_lazy_up(p_values[p], l_values[p], r - l + 1);
    }

    void propagate_lazy(int p, int l, int r)
    {
        lazy[p] = false;

        if (l == r)
            return;

        l_values[left(p)] = lazy[left(p)]
                                ? prop_lazy(l_values[left(p)], l_values[p])
                                : l_values[p];
        l_values[right(p)] = lazy[right(p)]
                                 ? prop_lazy(l_values[right(p)], l_values[p])
                                 : l_values[p];

        lazy[left(p)] = true;
        lazy[right(p)] = true;
    }

    void build(int p, int l, int r)
    {
        if (l == r)
        {
            p_values[p] = simple_node(l);
            return;
        }

        build(left(p), l, (l + r) / 2);
        build(right(p), (l + r) / 2 + 1, r);

        p_values[p] = prop(p_values[left(p)], p_values[right(p)]);
    }

    void set(int p, int l, int r, int i, int v)
    {
        if (lazy[p])
        {
            update_lazy(p, l, r);
            propagate_lazy(p, l, r);
        }

        if (l == r)
        {
            values[l] = v;
            p_values[p] = simple_node(l);
            return;
        }

        if (i <= (l + r) / 2)
            set(left(p), l, (l + r) / 2, i, v);
        else
            set(right(p), (l + r) / 2 + 1, r, i, v);

        p_values[p] = prop(p_values[left(p)], p_values[right(p)]);
    }

    int query(int p, int l, int r, int lq, int rq)
    {
        if (lazy[p])
        {
            update_lazy(p, l, r);
            propagate_lazy(p, l, r);
        }

        if (lq <= l && r <= rq)
            return p_values[p];

        int l1 = l, r1 = (l + r) / 2;
        int l2 = (l + r) / 2 + 1, r2 = r;

        if (l1 > rq || lq > r1)
            return query(right(p), l2, r2, lq, rq);
        if (l2 > rq || lq > r2)
            return query(left(p), l1, r1, lq, rq);

        int lt = query(left(p), l1, r1, lq, rq);
        int rt = query(right(p), l2, r2, lq, rq);

        return prop(lt, rt);
    }

    void set_rank(int p, int l, int r, int lq, int rq, int value)
    {
        if (lazy[p])
        {
            update_lazy(p, l, r);
            propagate_lazy(p, l, r);
        }

        if (l > rq || lq > r)
            return;

        if (lq <= l && r <= rq)
        {
            lazy[p] = true;
            l_values[p] = value;
            update_lazy(p, l, r);
            propagate_lazy(p, l, r);
            return;
        }

        set_rank(left(p), l, (l + r) / 2, lq, rq, value);
        set_rank(right(p), (l + r) / 2 + 1, r, lq, rq, value);

        p_values[p] = prop(p_values[left(p)], p_values[right(p)]);
    }

    int get(int p, int l, int r, int i)
    {
        if (lazy[p])
        {
            update_lazy(p, l, r);
            propagate_lazy(p, l, r);
        }

        if (l == r)
            return values[i];

        if (i <= (l + r) / 2)
            return get(left(p), l, (l + r) / 2, i);

        return get(right(p), (l + r) / 2 + 1, r, i);
    }

public:
    SegmentTreeLazy(vi &a)
    {
        values = a;
        n = a.size();
        p_values.assign(4 * n, 0);
        lazy.assign(4 * n, false);
        l_values.assign(4 * n, 0);
        build(1, 0, n - 1);
    }

    int query(int i, int j) { return query(1, 0, n - 1, i, j); }

    void set(int i, int v) { set(1, 0, n - 1, i, v); }

    void set_rank(int i, int j, int v) { set_rank(1, 0, n - 1, i, j, v); }

    int get(int i) { return get(1, 0, n - 1, i); }
};

\end{lstlisting}
\end{multicols}
Sparse Table
\begin{multicols}{2}
\begin{lstlisting}[language=C++]
class SparseTable
{

private:
    vector<vi> lookup;

    vi arr;

    int operation(int a, int b)
    {
        if (arr[a] <= arr[b])
            return a;

        return b;
    }

    int simple_node(int i) { return i; }

    void build_sparse_table()
    {
        int n = arr.size();

        for (int i = 0; i < n; i++)
            lookup[i][0] = simple_node(i);

        for (int j = 1; (1 << j) <= n; j++)
        {
            for (int i = 0; i <= n - (1 << j); i++)
                lookup[i][j] = operation(lookup[i][j - 1],
                                         lookup[i + (1 << (j - 1))][j - 1]);
        }
    }

public:
    SparseTable(vi &a)
    {
        int q = (int)log2(a.size());

        arr.assign(a.size(), 0);
        lookup.assign(a.size(), vi(q + 1));

        for (int i = 0; i < a.size(); i++)
            arr[i] = a[i];

        build_sparse_table();
    }

    int query(int l, int r)
    {
        int q = (int)log2(r - l + 1);

        return operation(lookup[l][q], lookup[r - (1 << q) + 1][q]);
    }

    int get(int i) { return arr[i]; }
};

\end{lstlisting}
\end{multicols}
Trie
\begin{multicols}{2}
\begin{lstlisting}[language=C++]
class Trie
{
private:
    int cant_string;
    int cant_string_me;
    int cant_node;
    char value;
    Trie *children[alphabet];

public:
    Trie(char a)
    {
        cant_string = 0;
        cant_node = 1;
        cant_string_me = 0;
        value = a;

        for (int i = 0; i < alphabet; i++)
            children[i] = nullptr;
    }

    pair<Trie *, int> search(string s)
    {
        Trie *node = this;
        int i = 0;

        while (node->children[s[i] - 'a'] != nullptr && i < s.size())
        {
            node = node->children[s[i] - 'a'];

            i++;
        }

        return {node, i};
    }

    void insert(string s)
    {
        int q = s.size() - search(s).second;

        Trie *node = this;

        for (int i = 0; i < s.size(); i++)
        {
            node->cant_node += q;

            if (node->children[s[i] - 'a'] == nullptr)
            {
                node->children[s[i] - 'a'] = new Trie(s[i]);
                q--;
            }

            node = node->children[s[i] - 'a'];
            node->cant_string_me++;
        }

        node->cant_string++;
    }

    void eliminate(string s)
    {
        if (!contains(s))
            return;

        Trie *node = this;
        int q = 0;

        for (int i = 0; i < s.size(); i++)
        {
            if (node->children[s[i] - 'a'] == nullptr)
            {
                node->children[s[i] - 'a'] = new Trie(s[i]);
            }

            if (node->children[s[i] - 'a']->cant_string_me == 1)
            {
                node->children[s[i] - 'a'] = nullptr;

                q = s.size() - i;
                break;
            }

            node = node->children[s[i] - 'a'];
            node->cant_string_me--;

            if (i == s.size() - 1)
                node->cant_string--;
        }

        node = this;

        for (int i = 0; i < s.size() - q + 1; i++)
        {
            node->cant_node -= q;
            node = node->children[s[i] - 'a'];
        }
    }

    bool contains(string s)
    {
        auto q = search(s);
        return q.second == s.size() && q.first->cant_string >= 1;
    }

    int cant_words_me() { return cant_string_me; }

    int cant_words() { return cant_string; }

    Trie *get(char a) { return children[a - 'a']; }

    int size() { return cant_node; }
};

\end{lstlisting}
\end{multicols}
Ufds
\begin{multicols}{2}
\begin{lstlisting}[language=C++]
class ufds
{
private:
    vector<int> p, rank, sizeSet;
    int disjoinSet;

public:
    ufds(int n)
    {
        p.assign(n, 0);
        rank.assign(n, 0);
        sizeSet.assign(n, 1);
        disjoinSet = n;
        for (int i = 0; i < n; i++)
        {
            p[i] = i;
        }
    }

    int find(int n)
    {
        if (n == p[n])
            return n;
        p[n] = find(p[n]);
        return p[n];
    }

    bool isSameSet(int i, int j) { return find(i) == find(j); }

    void unionSet(int i, int j)
    {
        if (!isSameSet(i, j))
        {
            disjoinSet--;
            int x = find(i);
            int y = find(j);
            if (rank[x] > rank[y])
            {
                p[y] = x;
                sizeSet[x] += sizeSet[y];
            }
            else
            {
                p[x] = y;
                sizeSet[y] += sizeSet[x];
                if (rank[x] == rank[y])
                    rank[y]++;
            }
        }
    }

    int numDisjoinset() { return disjoinSet; }

    int sizeofSet(int i) { return sizeSet[find(i)]; }
};

\end{lstlisting}
\end{multicols}


\end{document}