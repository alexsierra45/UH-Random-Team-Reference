\documentclass[leter]{amsart}
\usepackage[pdftex]{geometry}
\usepackage[utf8]{inputenc}

\usepackage{listings}
\usepackage{amssymb}
\usepackage{amsmath}
\usepackage{multicol}
\usepackage{courier}
\usepackage{fancyhdr}

\lstset{language=C++}
\lstset{columns=fullflexible}
\lstset{basicstyle=\scriptsize\ttfamily}
\lstset{showspaces=false}

\geometry{landscape}
\setlength{\topmargin}{-0.25in}
\setlength{\oddsidemargin}{0in}
\setlength{\evensidemargin}{0in}
\setlength{\columnsep}{1in}
\setlength{\columnseprule}{0.2pt}
\textwidth 9.5in

\pagestyle{fancy}
\lhead{UH-Random}
\chead{}
\rhead{\thepage}
\lfoot{}
\cfoot{}
\rfoot{}

\newcommand{\includecpp}[2]{
  \begin{multicols}{2}
    [\subsection{#1}\ ]
    \begin{lstlisting}[language=C++]
        #2
    \end{lstlisting}
  \end{multicols}
}

\begin{document}

\begin{center} \textbf{ UH Random Team Reference } \end{center} \vspace*{50pt} % Añade un espacio vertical de 50 puntos

\begin{multicols}{2} \tableofcontents \end{multicols}

\enlargethispage*{\baselineskip} \pagebreak

\section{TEMPLATE}
\begin{multicols}{2}
[\subsection{MAIN}\ ]
\begin{lstlisting}[language=C++]
#include <bits/stdc++.h>

#define MAX(a, b) (a > b) ? a : b
#define MIN(a, b) (a < b) ? a : b
#define int long long
#define vi vector<int>
#define pii pair<int, int>
#define vii vector<pii>

using namespace std;

void solve()
{
}

int32_t main()
{
    ios_base::sync_with_stdio(0);
    cin.tie(0);

    int t;
    cin >> t;

    for (int i = 0; i < t; i++)
    {
        solve();
    }

    return 0;
}

\end{lstlisting}
\end{multicols}
\enlargethispage*{\baselineskip}\section{MATH}
\begin{multicols}{2}
[\subsection{DISCRETE LOG}\ ]
\begin{lstlisting}[language=C++]

int powmod(int a, int b, int m)
{
    int res = 1;
    while (b > 0)
    {
        if (b & 1)
        {
            res = (res * 1ll * a) % m;
        }
        a = (a * 1ll * a) % m;
        b >>= 1;
    }
    return res;
}

int discrete_log(int a, int b, int m)
{
    a %= m, b %= m;
    int n = sqrt(m) + 1;
    map<int, int> vals;
    for (int p = 1; p <= n; ++p)
        vals[powmod(a, p * n, m)] = p;
    for (int q = 0; q <= n; ++q)
    {
        int cur = (powmod(a, q, m) * 1ll * b) % m;
        if (vals.count(cur))
        {
            int ans = vals[cur] * n - q;
            return ans;
        }
    }
    return -1;
}


\end{lstlisting}
\end{multicols}
\begin{multicols}{2}
[\subsection{NTT}\ ]
\begin{lstlisting}[language=C++]
using ll = long long;

const ll mod = (119 << 23) + 1, root = 62; // 998244353

ll qp(ll b, ll e)
{
    ll ans = 1;
    for (; e; b = b * b % mod, e /= 2)
        if (e & 1)
            ans = ans * b % mod;
    return ans;
}

void ntt(vector<ll> &a, vector<ll> &rt, vector<ll> &rev, int n)
{
    for (int i = 0; i < n; i++)
        if (i < rev[i])
            swap(a[i], a[rev[i]]);

    for (int k = 1; k < n; k *= 2)
        for (int i = 0; i < n; i += 2 * k)
            for (int j = 0; j < k; j++)
            {
                ll z = rt[j + k] * a[i + j + k] % mod, &ai = a[i + j];
                a[i + j + k] = (z > ai ? ai - z + mod : ai - z);
                ai += (ai + z >= mod ? z - mod : z);
            }
}

vector<ll> convolve(const vector<ll> &a, const vector<ll> &b)
{
    if (a.empty() || b.empty())
        return {};

    int s = a.size() + b.size() - 1, B = 32 - __builtin_clz(s), n = 1 << B;

    vector<ll> L(a), R(b), out(n), rt(n, 1), rev(n);
    L.resize(n), R.resize(n);

    for (int i = 0; i < n; i++)
        rev[i] = (rev[i / 2] | (i & 1) << B) / 2;

    ll curL = mod / 2, inv = qp(n, mod - 2);
    for (int k = 2; k < n; k *= 2)
    {
        ll z[] = {1, qp(root, curL /= 2)};
        for (int i = k; i < 2 * k; i++)
            rt[i] = rt[i / 2] * z[i & 1] % mod;
    }

    ntt(L, rt, rev, n);
    ntt(R, rt, rev, n);

    for (int i = 0; i < n; i++)
        out[-i & (n - 1)] = L[i] * R[i] % mod * inv % mod;

    ntt(out, rt, rev, n);

    return {out.begin(), out.begin() + s};
}


\end{lstlisting}
\end{multicols}
\begin{multicols}{2}
[\subsection{MATRIX POW}\ ]
\begin{lstlisting}[language=C++]
const int MAXN = 2;

struct Matrix
{
    ll mat[MAXN][MAXN];
};

Matrix operator*(const Matrix &a, const Matrix &b)
{
    Matrix c;
    for (int i = 0; i < MAXN; ++i)
        for (int j = 0; j < MAXN; ++j)
            c.mat[i][j] = 0;

    for (int i = 0; i < MAXN; i++)
    {
        for (int k = 0; k < MAXN; k++)
        {
            if (a.mat[i][k] == 0)
                continue;
            for (int j = 0; j < MAXN; j++)
            {
                c.mat[i][j] += a.mat[i][k] * b.mat[k][j];
            }
        }
    }
    return c;
}
Matrix operator^(Matrix &base, ll e)
{
    Matrix c;
    for (int i = 0; i < MAXN; i++)
        for (int j = 0; j < MAXN; j++)
            c.mat[i][j] = (i == j);
    while (e)
    {
        if (e & 1ll)
            c = c * base;
        base = base * base;
        e >>= 1;
    }
    return c;
}

\end{lstlisting}
\end{multicols}
\begin{multicols}{2}
[\subsection{FFT}\ ]
\begin{lstlisting}[language=C++]
struct point
{
    double x, y;
    point(double x = 0, double y = 0) : x(x), y(y) {}
};

point operator+(const point &a, const point &b)
{
    return {a.x + b.x, a.y + b.y};
}
point operator-(const point &a, const point &b)
{
    return {a.x - b.x, a.y - b.y};
}
point operator*(const point &a, const point &b)
{
    return {a.x * b.x - a.y * b.y, a.x * b.y + a.y * b.x};
}
point operator/(const point &a, double d) { return {a.x / d, a.y / d}; }

void fft(vector<point> &a, int sign = 1)
{
    int n = a.size(); // n should be a power of two
    double theta = 8 * sign * atan(1.0) / n;
    for (int i = 0, j = 1; j < n - 1; ++j)
    {
        for (int k = n >> 1; k > (i ^= k); k >>= 1)
            ;
        if (j < i)
            swap(a[i], a[j]);
    }
    for (int m, mh = 1; (m = mh << 1) <= n; mh = m)
    {
        int irev = 0;
        for (int i = 0; i < n; i += m)
        {
            point w = point(cos(theta * irev), sin(theta * irev));
            for (int k = n >> 2; k > (irev ^= k); k >>= 1)
                ;
            for (int j = i; j < mh + i; ++j)
            {
                int k = j + mh;
                point x = a[j] - a[k];
                a[j] = a[j] + a[k];
                a[k] = w * x;
            }
        }
    }
    if (sign == -1)
        for (auto &p : a)
            p = p / n;
}

vector<point> convolve(vector<point> &a, vector<point> &b)
{
    int n = a.size();
    int m = b.size();
    int k = n + m;
    while (k != (k & -k))
        k += (k & -k);
    while (a.size() < k)
        a.push_back(point(0, 0));
    while (b.size() < k)
        b.push_back(point(0, 0));

    fft(a, 1);
    fft(b, 1);

    vector<point> c(k);
    for (int i = 0; i < k; i++)
        c[i] = a[i] * b[i];

    fft(c, -1);

    return c;
}


\end{lstlisting}
\end{multicols}
\begin{multicols}{2}
[\subsection{PRIMITIVE ROOT}\ ]
\begin{lstlisting}[language=C++]
int powmod(int a, int b, int p)
{
    int res = 1;
    while (b)
        if (b & 1)
            res = int32_t(res * 1ll * a % p), --b;
        else
            a = int32_t(a * 1ll * a % p), b >>= 1;
    return res;
}

int generator(int p)
{
    vector<int> fact;
    int phi = p - 1, n = phi;
    for (int i = 2; i * i <= n; ++i)
        if (n % i == 0)
        {
            fact.push_back(i);
            while (n % i == 0)
                n /= i;
        }
    if (n > 1)
        fact.push_back(n);

    for (int res = 2; res <= p; ++res)
    {
        bool ok = true;
        for (size_t i = 0; i < fact.size() && ok; ++i)
            ok &= powmod(res, phi / fact[i], p) != 1;
        if (ok)
            return res;
    }
    return -1;
}

\end{lstlisting}
\end{multicols}
\begin{multicols}{2}
[\subsection{FWHT}\ ]
\begin{lstlisting}[language=C++]

using ll = long long;

const int mod = 1e9 + 7;

template <const int _mod_>
struct mod_int
{
    static const int mod = _mod_;
    int val;

    mod_int(long long v = 0)
    {
        if (v < 0)
            v = v % mod + mod;
        if (v >= mod)
            v %= mod;
        val = v;
    }

    static int mod_inv(int a, int m = mod)
    {
        int g = m, r = a, x = 0, y = 1;
        while (r != 0)
        {
            int q = g / r;
            g %= r;
            swap(g, r);
            x -= q * y;
            swap(x, y);
        }
        return x < 0 ? x + m : x;
    }

    explicit operator int() const { return val; }

    mod_int &operator+=(const mod_int &other)
    {
        val += other.val;
        if (val >= mod)
            val -= mod;
        return *this;
    }

    mod_int &operator-=(const mod_int &other)
    {
        val -= other.val;
        if (val < 0)
            val += mod;
        return *this;
    }

    static unsigned fast_mod(uint64_t x, unsigned m = mod)
    {
#if !defined(_WIN32) || defined(_WIN64)
        return x % m;
#endif
        // Optimized mod for Codeforces 32-bit machines.
        // x must be less than 2^32 * m for this to work, so that x / m fits in
        // a 32-bit integer.
        unsigned x_high = x >> 32, x_low = (unsigned)x;
        unsigned quot, rem;
        asm("divl %4\n"
            : "=a"(quot), "=d"(rem)
            : "d"(x_high), "a"(x_low), "r"(m));
        return rem;
    }

    mod_int &operator*=(const mod_int &other)
    {
        val = fast_mod((uint64_t)val * other.val);
        return *this;
    }

    mod_int &operator/=(const mod_int &other) { return *this *= other.inv(); }

    friend mod_int operator+(const mod_int &a, const mod_int &b)
    {
        return mod_int(a) += b;
    }
    friend mod_int operator-(const mod_int &a, const mod_int &b)
    {
        return mod_int(a) -= b;
    }
    friend mod_int operator*(const mod_int &a, const mod_int &b)
    {
        return mod_int(a) *= b;
    }
    friend mod_int operator/(const mod_int &a, const mod_int &b)
    {
        return mod_int(a) /= b;
    }

    mod_int &operator++()
    {
        val = val == mod - 1 ? 0 : val + 1;
        return *this;
    }

    mod_int &operator--()
    {
        val = val == 0 ? mod - 1 : val - 1;
        return *this;
    }

    mod_int operator++(int)
    {
        mod_int a = *this;
        ++*this;
        return a;
    }
    mod_int operator--(int)
    {
        mod_int a = *this;
        --*this;
        return a;
    }
    mod_int operator-() const { return val == 0 ? 0 : mod - val; }
    mod_int inv() const { return mod_inv(val); }

    bool operator==(const mod_int &other) const { return val == other.val; }
    bool operator!=(const mod_int &other) const { return val != other.val; }
    bool operator<(const mod_int &other) const { return val < other.val; }
    bool operator>(const mod_int &other) const { return val > other.val; }

    template <typename T>
    bool operator<(const T &other) const
    {
        return val < other;
    }

    template <typename T>
    bool operator>(const T &other) const
    {
        return val > other;
    }

    friend string to_string(const mod_int &m) { return to_string(m.val); }
    friend mod_int abs(const mod_int &m) { return mod_int(m.val); }

    friend ostream &operator<<(ostream &stream, const mod_int &m)
    {
        return stream << m.val;
    }
    friend istream &operator>>(istream &stream, mod_int &m)
    {
        return stream >> m.val;
    }
};

// Notar que se uso este tipo de datos entero, que implementa todas las
// operaciones basicos en el sistem residual modulo 10^9+7 para hacer mas
// faciles las implementaciones si se quiere implementarlo, se puede dejar en
// long long, y modificar las operaciones para mantener los resultados teneindo
// en cuenta el modulo a usar.
using T = mod_int<mod>;

void HADAMARD_XOR(vector<T> &a, bool inverse = false)
{
    int n = a.size();
    for (int k = 1; k < n; k <<= 1)
    {
        for (int i = 0; i < n; i += 2 * k)
        {
            for (int j = 0; j < k; j++)
            {
                T x = a[i + j];
                T y = a[i + j + k];
                a[i + j] = x + y;
                a[i + j + k] = x - y;
            }
        }
    }

    if (inverse)
    {
        T q = 1 / static_cast<T>(n);

        for (int i = 0; i < n; i++)
        {
            a[i] *= q;
        }
    }
}

void HADAMARD_AND(vector<T> &a, bool inverse = false)
{
    int n = a.size();
    for (int k = 1; k < n; k <<= 1)
    {
        for (int i = 0; i < n; i += 2 * k)
        {
            for (int j = 0; j < k; j++)
            {
                T x = a[i + j];
                T y = a[i + j + k];
                if (inverse)
                {
                    a[i + j] = -x + y;
                    a[i + j + k] = x;
                }
                else
                {
                    a[i + j] = y;
                    a[i + j + k] = x + y;
                }
            }
        }
    }
}

void HADAMARD_OR(vector<T> &a, bool inverse = false)
{
    int n = a.size();
    for (int k = 1; k < n; k <<= 1)
    {
        for (int i = 0; i < n; i += 2 * k)
        {
            for (int j = 0; j < k; j++)
            {
                T x = a[i + j];
                T y = a[i + j + k];
                if (inverse)
                {
                    a[i + j] = y;
                    a[i + j + k] = x - y;
                }
                else
                {
                    a[i + j] = x + y;
                    a[i + j + k] = x;
                }
            }
        }
    }
}

// Las demas operaciones a nivel de bit tienen una implementacion semejante
template <typename T>
vector<T> FWHT_XOR(vector<T> a, vector<T> b)
{
    bool eq = (a == b);
    int n = 1;

    while (n < (int)max(a.size(), b.size()))
    {
        n <<= 1;
    }

    a.resize(n);
    b.resize(n);

    HADAMARD_XOR(a);

    if (eq)
        b = a;
    else
        HADAMARD_XOR(b);

    for (int i = 0; i < n; i++)
    {
        a[i] *= b[i];
    }

    HADAMARD_XOR(a, true);

    return a;
}


\end{lstlisting}
\end{multicols}
\begin{multicols}{2}
[\subsection{EXTENDED EUCLIDEAN ALGORITHM}\ ]
\begin{lstlisting}[language=C++]
int gcd(int a, int b, int &x, int &y)
{
    x = 1, y = 0;
    int x1 = 0, y1 = 1, a1 = a, b1 = b;
    while (b1)
    {
        int q = a1 / b1;
        tie(x, x1) = make_tuple(x1, x - q * x1);
        tie(y, y1) = make_tuple(y1, y - q * y1);
        tie(a1, b1) = make_tuple(b1, a1 - q * b1);
    }
    return a1;
}

\end{lstlisting}
\end{multicols}
\enlargethispage*{\baselineskip}\section{GEOMETRY}
\begin{multicols}{2}
[\subsection{BASICS}\ ]
\begin{lstlisting}[language=C++]

db DEG_to_RAD(db d) { return d*M_PI / 180.0; }
db RAD_to_DEG(db r) { return r*180.0 / M_PI; }

db EPS = 1e-9;

struct point {
    db x, y;
    point() { x = y = 0.0; }
    point(db _x, db _y) : x(_x), y(_y) {}

    bool operator < (const point &other) const { 
        if (fabs(x-other.x) > EPS) 
        return x < other.x;
        return y < other.y; 
    }

    bool operator == (const point &other) const {
        return (fabs(x-other.x) < EPS) && (fabs(y-other.y) < EPS);
    }

    db dist(const point &other) { 
        return hypot(x-other.x, y-other.y);
    }

    point rotate(db theta) { 
        db rad = DEG_to_RAD(theta);
        return point(x*cos(rad) - y*sin(rad), x*sin(rad) + y*cos(rad));
    }
};

struct line {
    db a, b, c;
    line() {}
    line(db _a, db _b, db _c) : a(_a), b(_b), c(_c) {}

    void pointsToLine(const point &p1, const point &p2) {
        if (fabs(p1.x-p2.x) < EPS) {
            a = 1.0;
            b = 0.0;
            c = -p1.x;
        }
        else {
            a = -(db)(p1.y-p2.y) / (p1.x-p2.x);
            b = 1.0;
            c = -(db)(a*p1.x) - p1.y;
        }
    }

    void pointSlopeToLine(point p, db m) {
        a = -m; 
        b = 1.0; 
        c = -((a * p.x) + (b * p.y)); 
    }

    bool areParallel(const line &other) {
        return (fabs(a-other.a) < EPS) && (fabs(b-other.b) < EPS);
    }

    bool areSame(const line &other) {
        return areParallel(other) && (fabs(c-other.c) < EPS);
    }

    bool areIntersect(const line &other, point &p) {
        if (areParallel(other)) return false;
        p.x = (other.b*c - b*other.c) / (other.a*b - a*other.b);
        if (fabs(b) > EPS) p.y = -(a*p.x + c);
        else p.y = -(other.a*p.x + other.c);
        return true;
    }
};

struct vec{
    db x, y;
    vec(db _x, db _y) : x(_x), y(_y) {}
    vec(const point &a, const point &b) : x(b.x - a.x), y(b.y - a.y) {}

    vec scale(db s) {
        return vec(x*s, y*s);
    }

    point translate(const point &p) {
        return point(x+p.x, y+p.y);
    }

    db dot(vec a, vec b) { return a.x*b.x + a.y*b.y; }

    db norm_sq(vec v) { return v.x*v.x + v.y*v.y; }

    db angle(const point &a, const point &o, const point &b) {
        vec oa = vec(o, a), ob = vec(o, b); 
        return acos(dot(oa, ob) / sqrt(norm_sq(oa) * norm_sq(ob)));
    }

    db cross(vec a, vec b) { return a.x*b.y - a.y*b.x; }

    bool ccw(point p, point q, point r) {
        return cross(vec(p, q), vec(p, r)) > EPS;
    }

    bool collinear(point p, point q, point r) {
        return fabs(cross(vec(p, q), vec(p, r))) < EPS;
    }

    db distToLine(point p, point a, point b) {
        vec ap = vec(a, p), ab = vec(a, b);
        db u = dot(ap, ab) / norm_sq(ab);
        point c = ab.scale(u).translate(a);
        return c.dist(p);
    }

    db distToLineSegment(point p, point a, point b) {
        vec ap = vec(a, p), ab = vec(a, b);
        db u = dot(ap, ab) / norm_sq(ab);
        if (u < 0.0) {
            point c = point(a.x, a.y);
            return c.dist(p);
        }
        if (u > 1.0) {
            point c = point(b.x, b.y);
            return c.dist(p);
        }
        return distToLine(p, a, b); 
    }
};

struct circle {
    point c;
    db r;
    circle(const point &_c, db _r) : c(_c), r(_r) {}

    int inside(const point &p) {
        db dist = c.dist(p);
        return dist < r ? 1 : (fabs(dist-r) < EPS ? 0 : -1);
    }

    point inCircle(point p1, point p2, point p3) {
        line l1, l2;
        double ratio = p1.dist(p2) / p1.dist(p3);
        point p = vec(p2, p3).scale(ratio / (1+ratio)).translate(p2);
        l1.pointsToLine(p1, p);
        ratio = p2.dist(p1) / p2.dist(p3);
        p = vec(p1, p3).scale(ratio / (1+ratio)).translate(p1);
        l2.pointsToLine(p2, p);
        point c;
        l1.areIntersect(l2, c);
        return c;
    }
};


\end{lstlisting}
\end{multicols}
\begin{multicols}{2}
[\subsection{POLYGON}\ ]
\begin{lstlisting}[language=C++]

struct polygon {
    vector<point> P;
    polygon(const vector<point> &_P) : P(_P) {}

    db perimeter() {
        db ans = 0.0;
        for (int i = 0; i < (int)P.size()-1; ++i)
            ans += P[i].dist(P[i+1]);
        return ans;
    }

    db area() {
        db ans = 0.0;
        for (int i = 0; i < (int)P.size()-1; ++i)
            ans += (P[i].x*P[i+1].y - P[i+1].x*P[i].y);
        return fabs(ans) / 2.0;
    }

    bool isConvex(const vector<point> &P) {
        int n = (int)P.size();
        if (n <= 3) return false;
        bool firstTurn = vec().ccw(P[0], P[1], P[2]);
        for (int i = 1; i < n-1; ++i)
            if (vec().ccw(P[i], P[i+1], P[(i+2) == n ? 1 : i+2]) != firstTurn)
                return false;
        return true; 
    }

    int insidePolygon(point pt) {
        int n = (int)P.size();
        if (n <= 3) return -1;
        bool on_polygon = false;
        for (int i = 0; i < n-1; ++i)
            if (fabs(pt.dist(P[i]) + pt.dist(P[i+1]) - P[i].dist(P[i+1])) < EPS)
                on_polygon = true;
        if (on_polygon) return 0;
        double sum = 0.0;
        for (int i = 0; i < n-1; ++i) {
        if (vec().ccw(pt, P[i], P[i+1]))
            sum += vec().angle(P[i], pt, P[i+1]);
        else
            sum -= vec().angle(P[i], pt, P[i+1]);
        }
        return fabs(sum) > M_PI ? 1 : -1;
    }
};


\end{lstlisting}
\end{multicols}
\enlargethispage*{\baselineskip}\section{DATA STRUCTURE}
\begin{multicols}{2}
[\subsection{SEGMENT TREE}\ ]
\begin{lstlisting}[language=C++]
class SegmentTree
{
private:
    vi values;

    vi p_values;
    int n;

    int left(int p) { return p << 1; };

    int right(int p) { return (p << 1) + 1; }

    int simple_node(int index) { return values[index]; }

    int prop(int x, int y) { return x + y; }

    void build(int p, int l, int r)
    {
        if (l == r)
        {
            p_values[p] = simple_node(l);
            return;
        }

        build(left(p), l, (l + r) / 2);
        build(right(p), (l + r) / 2 + 1, r);

        p_values[p] = prop(p_values[left(p)], p_values[right(p)]);
    }

    void set(int p, int l, int r, int i, int v)
    {
        if (l == r)
        {
            values[l] = v;
            p_values[p] = simple_node(l);
            return;
        }

        if (i <= (l + r) / 2)
            set(left(p), l, (l + r) / 2, i, v);
        else
            set(right(p), (l + r) / 2 + 1, r, i, v);

        p_values[p] = prop(p_values[left(p)], p_values[right(p)]);
    }

    int query(int p, int l, int r, int lq, int rq)
    {
        if (lq <= l && r <= rq)
            return p_values[p];

        int l1 = l, r1 = (l + r) / 2;
        int l2 = (l + r) / 2 + 1, r2 = r;

        if (l1 > rq || lq > r1)
            return query(right(p), l2, r2, lq, rq);
        if (l2 > rq || lq > r2)
            return query(left(p), l1, r1, lq, rq);

        int lt = query(left(p), l1, r1, lq, rq);
        int rt = query(right(p), l2, r2, lq, rq);

        return prop(lt, rt);
    }

public:
    SegmentTree(vi &a)
    {
        values = a;
        n = a.size();
        p_values.assign(4 * n, 0);
        build(1, 0, n - 1);
    }

    int query(int i, int j) { return query(1, 0, n - 1, i, j); }

    void set(int i, int v) { set(1, 0, n - 1, i, v); }

    int get(int i) { return values[i]; }
};

\end{lstlisting}
\end{multicols}
\begin{multicols}{2}
[\subsection{DISJOINT SET UNION}\ ]
\begin{lstlisting}[language=C++]
 
struct dsu {
    vi p;
    void init(int n) {
        p = vi(n, -1);
    }
    int get(int x) {
        if (p[x] < 0)
            return x;
        return p[x] = get(p[x]);
    }
    void unite(int a, int b) {
        a = get(a);
        b = get(b);
        if (a != b) {
            if (p[a] > p[b])
                swap(a, b);
            p[a] += p[b];
            p[b] = a;
        }
    }
};


\end{lstlisting}
\end{multicols}
\begin{multicols}{2}
[\subsection{PBDS}\ ]
\begin{lstlisting}[language=C++]
#include <bits/extc++.h> // pbds
using namespace __gnu_pbds;
typedef tree<int, null_type, less<int>, rb_tree_tag,
             tree_order_statistics_node_update>
    ost;

int main()
{
  int n = 9;
  int A[] = {2, 4, 7, 10, 15, 23, 50, 65, 71}; // as in Chapter 2
  ost tree;
  for (int i = 0; i < n; ++i) // O(n log n)
    tree.insert(A[i]);
  // O(log n) select
  cout << *tree.find_by_order(0) << "\n";     // 1-smallest = 2
  cout << *tree.find_by_order(n - 1) << "\n"; // 9-smallest/largest = 71
  cout << *tree.find_by_order(4) << "\n";     // 5-smallest = 15
  // O(log n) rank
  cout << tree.order_of_key(2) << "\n";  // index 0 (rank 1)
  cout << tree.order_of_key(71) << "\n"; // index 8 (rank 9)
  cout << tree.order_of_key(15) << "\n"; // index 4 (rank 5)
  return 0;
}


\end{lstlisting}
\end{multicols}
\begin{multicols}{2}
[\subsection{SEGMENT TREE LAZY}\ ]
\begin{lstlisting}[language=C++]
class SegmentTreeLazy
{
private:
    vi values;
    vector<bool> lazy;
    vi l_values;
    vi p_values;
    int n;

    int left(int p) { return p << 1; };

    int right(int p) { return (p << 1) + 1; }

    int simple_node(int index) { return values[index]; }

    int prop(int x, int y) { return x + y; }

    int prop_lazy(int x, int y) { return x + y; }

    int prop_lazy_up(int x, int y, int s) { return x + y * s; }

    void update_lazy(int p, int l, int r)
    {
        if (l == r)
        {
            values[l] = prop_lazy(values[l], l_values[p]);
        }

        p_values[p] = prop_lazy_up(p_values[p], l_values[p], r - l + 1);
    }

    void propagate_lazy(int p, int l, int r)
    {
        lazy[p] = false;

        if (l == r)
            return;

        l_values[left(p)] = lazy[left(p)]
                                ? prop_lazy(l_values[left(p)], l_values[p])
                                : l_values[p];
        l_values[right(p)] = lazy[right(p)]
                                 ? prop_lazy(l_values[right(p)], l_values[p])
                                 : l_values[p];

        lazy[left(p)] = true;
        lazy[right(p)] = true;
    }

    void build(int p, int l, int r)
    {
        if (l == r)
        {
            p_values[p] = simple_node(l);
            return;
        }

        build(left(p), l, (l + r) / 2);
        build(right(p), (l + r) / 2 + 1, r);

        p_values[p] = prop(p_values[left(p)], p_values[right(p)]);
    }

    void set(int p, int l, int r, int i, int v)
    {
        if (lazy[p])
        {
            update_lazy(p, l, r);
            propagate_lazy(p, l, r);
        }

        if (l == r)
        {
            values[l] = v;
            p_values[p] = simple_node(l);
            return;
        }

        if (i <= (l + r) / 2)
            set(left(p), l, (l + r) / 2, i, v);
        else
            set(right(p), (l + r) / 2 + 1, r, i, v);

        p_values[p] = prop(p_values[left(p)], p_values[right(p)]);
    }

    int query(int p, int l, int r, int lq, int rq)
    {
        if (lazy[p])
        {
            update_lazy(p, l, r);
            propagate_lazy(p, l, r);
        }

        if (lq <= l && r <= rq)
            return p_values[p];

        int l1 = l, r1 = (l + r) / 2;
        int l2 = (l + r) / 2 + 1, r2 = r;

        if (l1 > rq || lq > r1)
            return query(right(p), l2, r2, lq, rq);
        if (l2 > rq || lq > r2)
            return query(left(p), l1, r1, lq, rq);

        int lt = query(left(p), l1, r1, lq, rq);
        int rt = query(right(p), l2, r2, lq, rq);

        return prop(lt, rt);
    }

    void set_rank(int p, int l, int r, int lq, int rq, int value)
    {
        if (lazy[p])
        {
            update_lazy(p, l, r);
            propagate_lazy(p, l, r);
        }

        if (l > rq || lq > r)
            return;

        if (lq <= l && r <= rq)
        {
            lazy[p] = true;
            l_values[p] = value;
            update_lazy(p, l, r);
            propagate_lazy(p, l, r);
            return;
        }

        set_rank(left(p), l, (l + r) / 2, lq, rq, value);
        set_rank(right(p), (l + r) / 2 + 1, r, lq, rq, value);

        p_values[p] = prop(p_values[left(p)], p_values[right(p)]);
    }

    int get(int p, int l, int r, int i)
    {
        if (lazy[p])
        {
            update_lazy(p, l, r);
            propagate_lazy(p, l, r);
        }

        if (l == r)
            return values[i];

        if (i <= (l + r) / 2)
            return get(left(p), l, (l + r) / 2, i);

        return get(right(p), (l + r) / 2 + 1, r, i);
    }

public:
    SegmentTreeLazy(vi &a)
    {
        values = a;
        n = a.size();
        p_values.assign(4 * n, 0);
        lazy.assign(4 * n, false);
        l_values.assign(4 * n, 0);
        build(1, 0, n - 1);
    }

    int query(int i, int j) { return query(1, 0, n - 1, i, j); }

    void set(int i, int v) { set(1, 0, n - 1, i, v); }

    void set_rank(int i, int j, int v) { set_rank(1, 0, n - 1, i, j, v); }

    int get(int i) { return get(1, 0, n - 1, i); }
};

\end{lstlisting}
\end{multicols}
\begin{multicols}{2}
[\subsection{AVL}\ ]
\begin{lstlisting}[language=C++]

struct avl {
    int key;
    int height;
    int size;
    avl *left;
    avl *right;

    avl(int k) {
        key = k;
        height = 1;
        size = 1;
        left = NULL;
        right = NULL;
    }

    int getBalance() {
        int leftHeight = 0;
        int rightHeight = 0;

        if (left != NULL)
            leftHeight = left->height;

        if (right != NULL)
            rightHeight = right->height;

        return leftHeight - rightHeight;
    }

    void updateSize() {
        int leftSize = 0;
        int rightSize = 0;

        if (left != NULL)
            leftSize = left->size;

        if (right != NULL)
            rightSize = right->size;

        size = leftSize + rightSize + 1;
    }

    void updateHeight() {
        int leftHeight = 0;
        int rightHeight = 0;

        if (left != NULL)
            leftHeight = left->height;

        if (right != NULL)
            rightHeight = right->height;

        height = max(leftHeight, rightHeight) + 1;
    }

    avl *rotateLeft() {
        avl *newRoot = right;
        right = newRoot->left;
        newRoot->left = this;
        updateHeight();
        newRoot->updateHeight();
        return newRoot;
    }

    avl *rotateRight() {
        avl *newRoot = left;
        left = newRoot->right;
        newRoot->right = this;
        updateHeight();
        newRoot->updateHeight();
        return newRoot;
    }

    avl *balance() {
        updateHeight();
        updateSize();
        int balance = getBalance();

        if (balance == 2) {
            if (left->getBalance() < 0)
                left = left->rotateLeft();
            return rotateRight();
        }

        if (balance == -2) {
            if (right->getBalance() > 0)
                right = right->rotateRight();
            return rotateLeft();
        }

        return this;
    }

    avl *insert(int k) {
        if (k < key) {
            if (left == NULL)
                left = new avl(k);
            else
                left = left->insert(k);
        }
        else {
            if (right == NULL)
                right = new avl(k);
            else
                right = right->insert(k);
        }

        return balance();
    }

    avl *findMin() {
        if (left == NULL)
            return this;
        else
            return left->findMin();
    }

    avl *removeMin() {
        if (left == NULL)
            return right;
        left = left->removeMin();
        return balance();
    }

    avl *remove(int k) {
        if (k < key)
            left = left->remove(k);
        else if (k > key)
            right = right->remove(k);
        else {
            avl *leftChild = left;
            avl *rightChild = right;

            delete this;

            if (rightChild == NULL)
                return leftChild;

            avl *min = rightChild->findMin();
            min->right = rightChild->removeMin();
            min->left = leftChild;
            return min->balance();
        }

        return balance();
    }

    int getRank(int k) {
        if (k < key) {
            if (left == NULL)
                return 0;
            else
                return left->getRank(k);
        }
        else if (k > key) {
            if (right == NULL)
                return 1 + left->size;
            else
                return 1 + left->size + right->getRank(k);
        }
        else
            return left->size;
    }

    int getKth(int k) {
        if (k < left->size)
            return left->getKth(k);
        else if (k > left->size)
            return right->getKth(k - left->size - 1);
        else
            return key;
    }

    static avl *join(avl *left, avl *right) {
        if (left->height < right->height) {
            right->left = join(left, right->left);
            return right->balance();
        }
        else if (left->height > right->height) {
            left->right = join(left->right, right);
            return left->balance();
        }
        else {
            avl *min = right->findMin();
            min->right = right->removeMin();
            min->left = left;
            return min->balance();
        }
    }

    pair<avl *, avl *> split(int k) {
        if (k < key) {
            pair<avl *, avl *> p = left->split(k);
            left = p.second;
            return {p.first, join(this, left)};
        }
        else {
            pair<avl *, avl *> p = right->split(k);
            right = p.first;
            return {join(this, right), p.second};
        }
    }
};


\end{lstlisting}
\end{multicols}
\begin{multicols}{2}
[\subsection{ABI}\ ]
\begin{lstlisting}[language=C++]
class Abi
{

private:
    vi p;
    int _size;

    int ls_one(int i) { return i & (-i); }

public:
    Abi(int n)
    {
        _size = n;
        p.assign(n + 1, 0);
    }

    int rsq(int k)
    {
        int sum = 0;

        for (int i = k; i > 0; i -= ls_one(i))
        {
            sum += p[i];
        }

        return sum;
    }

    int sum(int a, int b) { return rsq(b) - rsq(a - 1); }

    void adjust_sum(int k, int v)
    {
        for (int i = k; i < p.size(); i += ls_one(i))
            p[i] += v;
    }

    int size()
    {
        return _size;
    }
};

\end{lstlisting}
\end{multicols}
\begin{multicols}{2}
[\subsection{MOS}\ ]
\begin{lstlisting}[language=C++]

void remove(int idx); // TODO: remove value at idx from data structure
void add(int idx);    // TODO: add value at idx from data structure
int get_answer();     // TODO: extract the current answer of the data structure

int block_size;

struct Query
{
    int l, r, idx;
    bool operator<(Query other) const
    {
        return make_pair(l / block_size, r) <
               make_pair(other.l / block_size, other.r);
    }
};

vector<int> mo_s_algorithm(vector<Query> queries)
{
    vector<int> answers(queries.size());
    sort(queries.begin(), queries.end());

    // TODO: initialize data structure

    int cur_l = 0;
    int cur_r = -1;
    // invariant: data structure will always reflect the range [cur_l, cur_r]
    for (Query q : queries)
    {
        while (cur_l > q.l)
        {
            cur_l--;
            add(cur_l);
        }
        while (cur_r < q.r)
        {
            cur_r++;
            add(cur_r);
        }
        while (cur_l < q.l)
        {
            remove(cur_l);
            cur_l++;
        }
        while (cur_r > q.r)
        {
            remove(cur_r);
            cur_r--;
        }
        answers[q.idx] = get_answer();
    }
    return answers;
}


\end{lstlisting}
\end{multicols}
\begin{multicols}{2}
[\subsection{SQRT DECOMPOSITION}\ ]
\begin{lstlisting}[language=C++]

struct sqd {
    int n;
    int b;
    vi a;
    vi bsum;
    sqd(vi &a) {
        n = a.size();
        b = sqrt(n);
        this->a = a;
        bsum.assign(b + 1, 0);
        for (int i = 0; i < n; i++)
            bsum[i / b] += a[i];
    }
    void update(int i, int v) {
        bsum[i / b] += v - a[i];
        a[i] = v;
    }
    int query(int l, int r) {
        int sum = 0;
        for (int i = l; i <= r; i++)
            if (i % b == 0 && i + b - 1 <= r) {
                sum += bsum[i / b];
                i += b - 1;
            } else
                sum += a[i];
        return sum;
    }
};


\end{lstlisting}
\end{multicols}
\enlargethispage*{\baselineskip}\section{STRING}
\begin{multicols}{2}
[\subsection{SUFFIX ARRAY}\ ]
\begin{lstlisting}[language=C++]
class SuffixArray
{

public:
    SuffixArray(string s)
    {
        n = s.size() + 1;
        s_value = s + "$";

        ra.assign(n, 0);
        sa.assign(n, 0);
        temp_ra.assign(n, 0);
        temp_sa.assign(n, 0);

        construct_sa();
        build_lcp();
    }

    int size() { return n; }

    int get_int(int i) { return sa[i]; }

    int cant_match(string p)
    {
        pii ans = matching(p);

        if (ans.first == -1 && ans.second == -1)
            return 0;

        return ans.second - ans.first + 1;
    }

    int get_lcp(int i) { return plcp[sa[i]]; }

    int cant_substr() { return v_cant_substr; }

    string get_str(int i) { return s_value.substr(sa[i], n - sa[i] - 1); }

private:
    string s_value;
    int n;
    int v_cant_substr;

    vi ra;
    vi sa;
    vi c;
    vi temp_ra;
    vi temp_sa;
    vi phi;
    vi plcp;

    void counting_sort(int k)
    {
        int sum = 0;
        int maxi = max((int)300, n);

        c.assign(maxi, 0);

        for (int i = 0; i < n; i++)
            c[i + k < n ? ra[i + k] : 0]++;

        for (int i = 0; i < maxi; i++)
        {
            int tx = c[i];
            c[i] = sum;
            sum += tx;
        }

        for (int i = 0; i < n; i++)
            temp_sa[c[sa[i] + k < n ? ra[sa[i] + k] : 0]++] = sa[i];

        for (int i = 0; i < n; i++)
            sa[i] = temp_sa[i];
    }

    void construct_sa()
    {
        int k, r;

        for (int i = 0; i < n; i++)
        {
            ra[i] = s_value[i];
            sa[i] = i;
        }

        for (k = 1; k < n; k <<= 1)
        {
            counting_sort(k);
            counting_sort(0);

            temp_ra[sa[0]] = r = 0;

            for (int i = 1; i < n; i++)
                temp_ra[sa[i]] = (ra[sa[i]] == ra[sa[i - 1]] && ra[sa[i] + k] == ra[sa[i - 1] + k]) ? r : ++r;

            for (int i = 0; i < n; i++)
                ra[i] = temp_ra[i];

            if (ra[sa[n - 1]] == n - 1)
                break;
        }
    }

    pii matching(string p)
    {
        int l = 0;
        int r = n - 1;
        int p_size = p.size();

        string comp;

        while (l < r)
        {
            int m = (l + r) / 2;

            comp = s_value.substr(sa[m], min(n - sa[m], p_size));

            if (comp >= p)
                r = m;
            else
                l = m + 1;
        }

        comp = s_value.substr(sa[l], min(n - sa[l], p_size));

        if (comp != p)
            return {-1, -1};

        int ans_l = l;

        l = 0;
        r = n - 1;

        while (l < r)
        {
            int m = (l + r) / 2;

            comp = s_value.substr(sa[m], min(n - sa[m], p_size));

            if (comp > p)
                r = m;
            else
                l = m + 1;
        }

        comp = s_value.substr(sa[r], min(n - sa[r], p_size));

        if (comp != p)
            r--;

        int ans_r = r;

        return {ans_l, ans_r};
    }

    void build_lcp()
    {
        phi.assign(n, 0);
        plcp.assign(n, 0);

        phi[0] = -1;

        for (int i = 1; i < n; i++)
            phi[sa[i]] = sa[i - 1];

        int l = 0;
        int q = 0;
        for (int i = 0; i < n; i++)
        {
            if (phi[i] == -1)
            {
                plcp[i] = 0;
                continue;
            }

            while (s_value[i + l] == s_value[phi[i] + l])
                l++;

            plcp[i] = l;
            q += l;
            l = max(l - 1, (int)0);
        }

        v_cant_substr = n * (n - 1) / 2 - q;
    }
};

\end{lstlisting}
\end{multicols}
\begin{multicols}{2}
[\subsection{Z FUNCTION}\ ]
\begin{lstlisting}[language=C++]
// Z[i] is the length of the longest substring
// starting from S[i] which is also a prefix of S.
vi z_function(string s)
{
    int n = (int)s.length();
    vi z(n);

    for (int i = 1, l = 0, r = 0; i < n; ++i)
    {
        if (i <= r)
            z[i] = min(r - i + 1, z[i - l]);
        while (i + z[i] < n && s[z[i]] == s[i + z[i]])
            ++z[i];
        if (i + z[i] - 1 > r)
            l = i, r = i + z[i] - 1;
    }
    return z;
}

// suff[i] = length of the longest common suffix of s and s[0..i]
vi suffixes(const string &s)
{
    int n = s.length();

    vi suff(n, n);

    for (int i = n - 2, g = n - 1, f; i >= 0; --i)
    {
        if (i > g && suff[i + n - 1 - f] != i - g)
            suff[i] = min(suff[i + n - 1 - f], i - g);
        else
        {
            for (g = min(g, f = i); g >= 0 && s[g] == s[g + n - 1 - f]; --g)
                ;
            suff[i] = f - g;
        }
    }

    return suff;
}

\end{lstlisting}
\end{multicols}
\begin{multicols}{2}
[\subsection{HASHING}\ ]
\begin{lstlisting}[language=C++]
struct custom_hash
{
    static uint64_t splitmix64(uint64_t x)
    {
        x += 0x9e3779b97f4a7c15;
        x = (x ^ (x >> 30)) * 0xbf58476d1ce4e5b9;
        x = (x ^ (x >> 27)) * 0x94d049bb133111eb;
        return x ^ (x >> 31);
    }

    size_t operator()(uint64_t x) const
    {
        static const uint64_t FIXED_RANDOM =
            chrono::steady_clock::now().time_since_epoch().count();
        return splitmix64(x + FIXED_RANDOM);
    }
};

const int MAXN = 5e5 + 5;
const int MOD = 1e9 + 7;
int mod_pow(int b, int e)
{
    int res = 1;
    while (e)
    {
        if (e & 1)
            res = res * b % MOD;
        b = b * b % MOD;
        e >>= 1;
    }
    return res % MOD;
}

typedef unordered_map<ll, ll, custom_hash> safe_map;

int dx[] = {1, 0, -1, 0, 1, -1, -1, 1};
int dy[] = {0, 1, 0, -1, 1, 1, -1, -1};

int dp[MAXN];
vi ady_list[MAXN];

int n, m, t, k;
int f[MAXN];
int comb(int n, int k)
{
    int fk = mod_pow(f[k], MOD - 2);
    int fnk = mod_pow(f[n - k], MOD - 2);
    int x = ((f[n] * fk) % MOD * fnk % MOD);

    return x;
}


\end{lstlisting}
\end{multicols}
\begin{multicols}{2}
[\subsection{KMP PF}\ ]
\begin{lstlisting}[language=C++]
vi prefix_function(string p)
{
    vi pf(p.size());

    pf[0] = 0;
    int k = 0;

    for (int i = 1; i < p.size(); i++)
    {
        while (k > 0 && p[k] != p[i])
            k = pf[k - 1];

        if (p[k] == p[i])
            k++;

        pf[i] = k;
    }

    return pf;
}

vi kmp(string t, string p)
{
    vi result;
    vi pf = prefix_function(p);
    int k = 0;

    for (int i = 0; i < t.size(); i++)
    {
        while (k > 0 && p[k] != t[i])
            k = pf[k - 1];

        if (p[k] == t[i])
            k++;

        if (k == p.size())
        {
            result.push_back(i - (p.size() - 1));
            k = pf[k - 1];
        }
    }

    return result;
}

\end{lstlisting}
\end{multicols}
\begin{multicols}{2}
[\subsection{TRIE}\ ]
\begin{lstlisting}[language=C++]
class Trie
{
private:
    int cant_string;
    int cant_string_me;
    int cant_node;
    char value;
    Trie *children[alphabet];

public:
    Trie(char a)
    {
        cant_string = 0;
        cant_node = 1;
        cant_string_me = 0;
        value = a;

        for (int i = 0; i < alphabet; i++)
            children[i] = NULL;
    }

    pair<Trie *, int> search(string s)
    {
        Trie *node = this;
        int i = 0;

        while (i < s.size() && node->children[s[i] - first_char] != NULL)
        {
            node = node->children[s[i] - first_char];

            i++;
        }

        return {node, i};
    }

    void insert(string s)
    {
        int q = s.size() - search(s).second;

        Trie *node = this;

        for (int i = 0; i < s.size(); i++)
        {
            node->cant_node += q;

            if (node->children[s[i] - first_char] == NULL)
            {
                node->children[s[i] - first_char] = new Trie(s[i]);
                q--;
            }

            node = node->children[s[i] - first_char];
            node->cant_string_me++;
        }

        node->cant_string++;
    }

    void eliminate(string s)
    {
        if (!contains(s))
            return;

        Trie *node = this;
        int q = 0;

        for (int i = 0; i < s.size(); i++)
        {
            if (node->children[s[i] - first_char] == NULL)
            {
                node->children[s[i] - first_char] = new Trie(s[i]);
            }

            if (node->children[s[i] - first_char]->cant_string_me == 1)
            {
                node->children[s[i] - first_char] = NULL;

                q = s.size() - i;
                break;
            }

            node = node->children[s[i] - first_char];
            node->cant_string_me--;

            if (i == s.size() - 1)
                node->cant_string--;
        }

        node = this;

        for (int i = 0; i < s.size() - q + 1; i++)
        {
            node->cant_node -= q;
            node = node->children[s[i] - first_char];
        }
    }

    bool contains(string s)
    {
        auto q = search(s);
        return q.second == s.size() && q.first->cant_string >= 1;
    }

    int cant_words_me() { return cant_string_me; }

    int cant_words() { return cant_string; }

    Trie *get(char a) { return children[a - first_char]; }

    int size() { return cant_node; }
};

\end{lstlisting}
\end{multicols}
\enlargethispage*{\baselineskip}\section{GRAPH}
\begin{multicols}{2}
[\subsection{TOPOLOGICAL SORT}\ ]
\begin{lstlisting}[language=C++]
vector<int> topoSort(int V, vector<int> adj[])
{
    vector<int> in(V);
    vector<int> resp;

    for (int i = 0; i < V; i++)
    {
        for (int j = 0; j < adj[i].size(); j++)
        {
            in[adj[i][j]]++;
        }
    }

    queue<int> q;

    for (int i = 0; i < V; i++)
    {
        if (in[i] == 0)
            q.push(i);
    }

    while (q.size() != 0)
    {
        int n = q.front();
        q.pop();

        for (int i = 0; i < adj[n].size(); i++)
        {
            in[adj[n][i]]--;

            if (in[adj[n][i]] == 0)
                q.push(adj[n][i]);
        }

        resp.push_back(n);
    }

    return resp;
}

\end{lstlisting}
\end{multicols}
\begin{multicols}{2}
[\subsection{DIJSKTRA}\ ]
\begin{lstlisting}[language=C++]
int infinite = (int)1e9;

// O(V^2)
vector<int> dijkstra1(int V, vector<vector<int>> adj[], int S)
{
    vector<int> d;

    d.assign(V, infinite);
    d[S] = 0;

    vector<bool> mask;

    mask.assign(V, false);

    for (int i = 0; i < V; i++)
    {
        int m = infinite;
        int act = -1;

        for (int j = 0; j < V; j++)
        {
            if (mask[j])
                continue;

            if (m > d[j])
            {
                m = d[j];
                act = j;
            }
        }

        for (int j = 0; j < adj[act].size(); j++)
        {
            if (d[act] + adj[act][j][1] < d[adj[act][j][0]])
            {
                d[adj[act][j][0]] = d[act] + adj[act][j][1];
            }
        }

        mask[act] = true;
    }

    return d;
}

// O((V+E)log(E))
vi dijkstra2(int V, vii adj[], int S)
{
    vector<int> d;

    d.assign(V, infinite);
    d[S] = 0;

    priority_queue<pair<int, int>> q;
    q.push({d[S], S});

    while (!q.empty())
    {
        int act = q.top().second;
        int m = abs(q.top().first);
        q.pop();

        if (m > d[act])
            continue;

        for (int j = 0; j < adj[act].size(); j++)
        {
            if (d[act] + adj[act][j].second < d[adj[act][j].first])
            {
                d[adj[act][j].first] = d[act] + adj[act][j].second;
                q.push({-d[adj[act][j].first], adj[act][j].first});
            }
        }
    }

    return d;
}

\end{lstlisting}
\end{multicols}
\begin{multicols}{2}
[\subsection{BRIDGE EDGES}\ ]
\begin{lstlisting}[language=C++]
vector<bool> visited;
vector<int> t;
vector<int> low;
set<pair<int, int>> bridges;

void dfs_bridges(vector<int> adj[], int n, int p, int q)
{
    t[n] = q;
    low[n] = q++;
    visited[n] = true;

    int j = 0;

    for (int i = 0; i < adj[n].size(); i++)
    {
        if (!visited[adj[n][i]])
        {
            dfs_bridges(adj, adj[n][i], n, q);
            low[n] = min(low[adj[n][i]], low[n]);
            j++;
        }
        else if (adj[n][i] != p)
        {
            low[n] = min(t[adj[n][i]], low[n]);
        }
    }

    if (t[n] == low[n] && p != -1)
    {
        bridges.insert({min(n, p), max(n, p)});
    }
}

set<pair<int, int>> bridge_edges(int V, vector<int> adj[])
{
    visited.assign(V, false);
    t.assign(V, -1);
    low.assign(V, -1);
    bridges = set<pair<int, int>>();

    for (int i = 0; i < V; i++)
    {
        if (!visited[i])
        {
            dfs_bridges(adj, i, -1, 1);
        }
    }

    return bridges;
}

\end{lstlisting}
\end{multicols}
\begin{multicols}{2}
[\subsection{PRIM}\ ]
\begin{lstlisting}[language=C++]
int spanningTreePrim(int V, vector<vector<int>> adj[])
{

    priority_queue<pair<int, int>> q;

    vector<bool> mask;
    mask.assign(V, false);
    mask[0] = true;

    int cost = 0;

    for (int i = 0; i < adj[0].size(); i++)
    {
        q.push({-adj[0][i][1], adj[0][i][0]});
    }

    while (q.size() != 0)
    {
        auto aux = q.top();
        q.pop();

        int k = aux.second;
        if (mask[k])
            continue;

        mask[k] = true;
        cost += abs(aux.first);

        for (int i = 0; i < adj[k].size(); i++)
        {
            if (!mask[adj[k][i][0]])
            {
                q.push({-adj[k][i][1], adj[k][i][0]});
            }
        }
    }

    return cost;
}

\end{lstlisting}
\end{multicols}
\begin{multicols}{2}
[\subsection{KRUSKAL}\ ]
\begin{lstlisting}[language=C++]
class ufds
{
private:
    vector<int> p, rank, sizeSet;
    int disjoinSet;

public:
    ufds(int n)
    {
        p.assign(n, 0);
        rank.assign(n, 0);
        sizeSet.assign(n, 1);
        disjoinSet = n;
        for (int i = 0; i < n; i++)
        {
            p[i] = i;
        }
    }

    int find(int n)
    {
        if (n == p[n])
            return n;
        p[n] = find(p[n]);
        return p[n];
    }

    bool isSameSet(int i, int j) { return find(i) == find(j); }

    void unionSet(int i, int j)
    {
        if (!isSameSet(i, j))
        {
            disjoinSet--;
            int x = find(i);
            int y = find(j);
            if (rank[x] > rank[y])
            {
                p[y] = x;
                sizeSet[x] += sizeSet[y];
            }
            else
            {
                p[x] = y;
                sizeSet[y] += sizeSet[x];
                if (rank[x] == rank[y])
                    rank[y]++;
            }
        }
    }

    int numDisjoinset() { return disjoinSet; }

    int sizeofSet(int i) { return sizeSet[find(i)]; }
};

// Function to find sum of weights of edges of the Minimum Spanning Tree.
int spanningTreeKruskal(int V, vector<vector<int>> adj[])
{
    ufds dsu(V);

    vector<pair<int, pair<int, int>>> a;

    for (int i = 0; i < V; i++)
    {
        for (int j = 0; j < adj[i].size(); j++)
        {
            a.push_back({adj[i][j][1], {i, adj[i][j][0]}});
        }
    }

    sort(a.begin(), a.end());

    int cost = 0;

    for (int i = 0; i < a.size(); i++)
    {
        if (!dsu.isSameSet(a[i].second.first, a[i].second.second))
        {
            cost += a[i].first;

            dsu.unionSet(a[i].second.first, a[i].second.second);
        }
    }

    return cost;
}

\end{lstlisting}
\end{multicols}
\begin{multicols}{2}
[\subsection{BELLMAN FORD}\ ]
\begin{lstlisting}[language=C++]
int infinite = (int)1e9;

vector<int> bellman_ford(int V, vector<vector<int>> &edges, int S)
{
    vector<int> d;
    d.assign(V, infinite);
    d[S] = 0;

    for (int i = 0; i < V - 1; i++)
    {
        for (int j = 0; j < edges.size(); j++)
        {
            if (d[edges[j][0]] + edges[j][2] < d[edges[j][1]])
            {
                d[edges[j][1]] = d[edges[j][0]] + edges[j][2];
            }
        }
    }

    for (int j = 0; j < edges.size(); j++)
    {
        if (d[edges[j][0]] + edges[j][2] < d[edges[j][1]])
        {
            vector<int> resp(1);
            resp[0] = -1;

            return resp;
        }
    }

    return d;
}

\end{lstlisting}
\end{multicols}
\begin{multicols}{2}
[\subsection{DINIC}\ ]
\begin{lstlisting}[language=C++]
template <typename flow_type>
struct dinic
{
    struct edge
    {
        size_t src, dst, rev;
        flow_type flow, cap;
    };

    int n;
    vector<vector<edge>> adj;

    dinic(int n) : n(n), adj(n), level(n), q(n), it(n) {}

    void add_edge(size_t src, size_t dst, flow_type cap, flow_type rcap = 0)
    {
        adj[src].push_back({src, dst, adj[dst].size(), 0, cap});
        if (src == dst)
            adj[src].back().rev++;
        adj[dst].push_back({dst, src, adj[src].size() - 1, 0, rcap});
    }

    vector<int> level, q, it;

    bool bfs(int source, int sink)
    {
        fill(level.begin(), level.end(), -1);
        for (int qf = level[q[0] = sink] = 0, qb = 1; qf < qb; ++qf)
        {
            sink = q[qf];
            for (edge &e : adj[sink])
            {
                edge &r = adj[e.dst][e.rev];
                if (r.flow < r.cap && level[e.dst] == -1)
                    level[q[qb++] = e.dst] = 1 + level[sink];
            }
        }
        return level[source] != -1;
    }

    flow_type augment(int source, int sink, flow_type flow)
    {
        if (source == sink)
            return flow;
        for (; it[source] != adj[source].size(); ++it[source])
        {
            edge &e = adj[source][it[source]];
            if (e.flow < e.cap && level[e.dst] + 1 == level[source])
            {
                flow_type delta = augment(e.dst, sink, min(flow, e.cap - e.flow));
                if (delta > 0)
                {
                    e.flow += delta;
                    adj[e.dst][e.rev].flow -= delta;
                    return delta;
                }
            }
        }
        return 0;
    }

    flow_type max_flow(int source, int sink)
    {
        for (int u = 0; u < n; ++u)
            for (edge &e : adj[u])
                e.flow = 0;
        flow_type flow = 0;
        flow_type oo = numeric_limits<flow_type>::max();

        while (bfs(source, sink))
        {
            fill(it.begin(), it.end(), 0);
            for (flow_type f; (f = augment(source, sink, oo)) > 0;)
                flow += f;

        } // level[u] = -1 => source side of min cut
        return flow;
    }
};

\end{lstlisting}
\end{multicols}
\begin{multicols}{2}
[\subsection{CENTROID DESCOMPOSITION}\ ]
\begin{lstlisting}[language=C++]
const int MAXN = 2e5 + 5;

vi ady[MAXN];

bitset<MAXN> is_centroid;

int sz[MAXN], ct_par[MAXN];
void centroid_dfs(int node, int parent)
{
    sz[node] = 1;
    for (int &nxt : ady[node])
    {
        if (is_centroid[nxt] || nxt == parent)
            continue;
        centroid_dfs(nxt, node);
        sz[node] += sz[nxt];
    }
}
int get_centroid(int node, int parent, int tree_sz)
{
    for (int nxt : ady[node])
    {
        if (is_centroid[nxt] || nxt == parent)
            continue;
        if (sz[nxt] * 2 > tree_sz)
            return get_centroid(nxt, node, tree_sz);
    }
    return node;
}
void centroid_decomp(int node, int parent = -1)
{
    centroid_dfs(node, -1);
    int tree_sz = sz[node];
    int centroid = get_centroid(node, -1, tree_sz);
    is_centroid[centroid] = 1;
    ct_par[centroid] = parent;

    for (int &child : ady[centroid])
    {
        if (is_centroid[child])
            continue;
        centroid_decomp(child, centroid);
    }
}

void solve()
{
}

int32_t main()
{
    ios_base::sync_with_stdio(0);
    cin.tie(0);

    int t;
    cin >> t;

    for (int i = 0; i < t; i++)
    {
        solve();
    }

    return 0;
}
\end{lstlisting}
\end{multicols}
\begin{multicols}{2}
[\subsection{LOWER BOUND FLOW}\ ]
\begin{lstlisting}[language=C++]

template <typename T>
struct dinic
{
    struct edge
    {
        int src, dst;
        T low, cap, flow;
        int rev;
    };

    int n;
    vector<vector<edge>> adj;

    dinic(int n) : n(n), adj(n + 2) {}

    void add_edge(int src, int dst, T low, T cap)
    {
        adj[src].push_back({src, dst, low, cap, 0, (int)adj[dst].size()});
        if (src == dst)
            adj[src].back().rev++;
        adj[dst].push_back({dst, src, 0, 0, 0, (int)adj[src].size() - 1});
    }

    vector<int> level, iter;

    T augment(int u, int t, T cur)
    {
        if (u == t)
            return cur;
        for (int &i = iter[u]; i < (int)adj[u].size(); ++i)
        {
            edge &e = adj[u][i];
            if (e.cap - e.flow > 0 && level[u] > level[e.dst])
            {
                T f = augment(e.dst, t, min(cur, e.cap - e.flow));
                if (f > 0)
                {
                    e.flow += f;
                    adj[e.dst][e.rev].flow -= f;
                    return f;
                }
            }
        }
        return 0;
    }

    int bfs(int s, int t)
    {
        level.assign(n + 2, n + 2);
        level[t] = 0;
        queue<int> Q;
        for (Q.push(t); !Q.empty(); Q.pop())
        {
            int u = Q.front();
            if (u == s)
                break;
            for (edge &e : adj[u])
            {
                edge &erev = adj[e.dst][e.rev];
                if (erev.cap - erev.flow > 0 && level[e.dst] > level[u] + 1)
                {
                    Q.push(e.dst);
                    level[e.dst] = level[u] + 1;
                }
            }
        }
        return level[s];
    }

    const T oo = numeric_limits<T>::max();

    T max_flow(int source, int sink)
    {
        vector<T> delta(n + 2);

        for (int u = 0; u < n; ++u) // initialize
            for (auto &e : adj[u])
            {
                delta[e.src] -= e.low;
                delta[e.dst] += e.low;
                e.cap -= e.low;
                e.flow = 0;
            }

        T sum = 0;
        int s = n, t = n + 1;

        for (int u = 0; u < n; ++u)
        {
            if (delta[u] > 0)
            {
                add_edge(s, u, 0, delta[u]);
                sum += delta[u];
            }
            else if (delta[u] < 0)
                add_edge(u, t, 0, -delta[u]);
        }

        add_edge(sink, source, 0, oo);
        T flow = 0;

        while (bfs(s, t) < n + 2)
        {
            iter.assign(n + 2, 0);
            for (T f; (f = augment(s, t, oo)) > 0;)
                flow += f;
        }

        if (flow != sum)
            return -1; // no solution

        for (int u = 0; u < n; ++u)
            for (auto &e : adj[u])
            {
                e.cap += e.low;
                e.flow += e.low;
                edge &erev = adj[e.dst][e.rev];
                erev.cap -= e.low;
                erev.flow -= e.low;
            }

        adj[sink].pop_back();
        adj[source].pop_back();

        while (bfs(source, sink) < n + 2)
        {
            iter.assign(n + 2, 0);
            for (T f; (f = augment(source, sink, oo)) > 0;)
                flow += f;
        } // level[u] == n + 2 ==> s-side

        return flow;
    }
};


\end{lstlisting}
\end{multicols}
\begin{multicols}{2}
[\subsection{FLOYD WARSHALL}\ ]
\begin{lstlisting}[language=C++]
int infinite = (int)1e8;

void shortest_distance(vector<vector<int>> &matrix)
{
    for (int k = 0; k < matrix.size(); k++)
    {
        for (int i = 0; i < matrix.size(); i++)
        {
            for (int j = 0; j < matrix[0].size(); j++)
            {
                matrix[i][j] = min(matrix[i][j], matrix[i][k] + matrix[k][j]);
            }
        }
    }
}

void find_path_k(vector<vector<bool>> &matrix, int k)
{
    for (int x = 0; x < k; x++)
    {
        for (int i = 0; i < matrix.size(); i++)
        {
            for (int j = 0; j < matrix[0].size(); j++)
            {
                matrix[i][j] = matrix[i][j] || (matrix[i][x] && matrix[x][j]);
            }
        }
    }
}

\end{lstlisting}
\end{multicols}
\begin{multicols}{2}
[\subsection{DFS BFS}\ ]
\begin{lstlisting}[language=C++]
void dfs_g(int n, int c, vi adj[], vector<bool> &visited, vi &cc)
{
    visited[n] = true;
    cc[n] = c;

    for (int i = 0; i < adj[n].size(); i++)
    {
        if (!visited[adj[n][i]])
            dfs_g(adj[n][i], c, adj, visited, cc);
    }
}

void dfs_t(int n, int p, int d, vi adj[], vi &deep)
{
    deep[n] = d;

    for (int i = 0; i < adj[n].size(); i++)
    {
        if (p != adj[n][i])
            dfs_t(adj[n][i], n, d + 1, adj, deep);
    }
}

vi bfs(int node, int n, vi adj[])
{
    vi result(n);
    vector<bool> visited;
    visited.assign(n, false);

    queue<int> q;
    visited[node] = true;

    q.push(node);

    while (q.size() != 0)
    {
        int w = q.front();
        q.pop();

        for (int i = 0; i < adj[w].size(); i++)
        {
            if (!visited[adj[w][i]])
            {
                q.push(adj[w][i]);
                result[adj[w][i]] = result[w] + 1;
                visited[adj[w][i]] = true;
            }
        }
    }

    return result;
}

\end{lstlisting}
\end{multicols}
\begin{multicols}{2}
[\subsection{SCC TARJANS}\ ]
\begin{lstlisting}[language=C++]
stack<int> q;
vector<bool> mask;
vector<int> cc_list;

void g_transp(int V, vector<int> adj[], vector<int> new_adj[])
{
    for (int i = 0; i < V; i++)
    {
        for (int j = 0; j < adj[i].size(); j++)
        {
            new_adj[adj[i][j]].push_back(i);
        }
    }
}

void dfs_visit(int n, vector<int> adj[], int cc)
{
    mask[n] = true;

    for (int i = 0; i < adj[n].size(); i++)
    {
        if (!mask[adj[n][i]])
            dfs_visit(adj[n][i], adj, cc);
    }

    if (cc == -1)
        q.push(n);
    else
    {
        cc_list[n] = cc;
    }
}

void tarjans(int V, vector<int> adj[])
{
    vector<int> new_adj[V];
    g_transp(V, adj, new_adj);

    mask.assign(V, false);
    cc_list.assign(V, -1);

    for (int i = 0; i < V; i++)
    {
        if (mask[i])
            continue;

        dfs_visit(i, adj, -1);
    }

    for (int i = 0; i < V; i++)
        mask[i] = false;

    int ind = 0;

    while (q.size() != 0)
    {
        int act = q.top();

        q.pop();

        if (!mask[act])
        {
            dfs_visit(act, new_adj, ind);
            ind++;
        }
    }
}

\end{lstlisting}
\end{multicols}
\begin{multicols}{2}
[\subsection{MIN COST MAX FLOW}\ ]
\begin{lstlisting}[language=C++]

template <typename flow_type, typename cost_type> struct min_cost_max_flow {
	struct edge {
		size_t src, dst, rev;
		flow_type flow, cap;
		cost_type cost;
	};

	int n;
	vector<vector<edge>> adj;

	min_cost_max_flow(int n) : n(n), adj(n), potential(n), dist(n), back(n) {}

	void add_edge(size_t src, size_t dst, flow_type cap, cost_type cost) {
		adj[src].push_back({src, dst, adj[dst].size(), 0, cap, cost});
		if (src == dst)
			adj[src].back().rev++;
		adj[dst].push_back({dst, src, adj[src].size() - 1, 0, 0, -cost});
	}

	vector<cost_type> potential;

	inline cost_type rcost(const edge &e) {
		return e.cost + potential[e.src] - potential[e.dst];
	}

	void bellman_ford(int source) {
		for (int k = 0; k < n; ++k)
			for (int u = 0; u < n; ++u)
				for (edge &e : adj[u])
					if (e.cap > 0 && rcost(e) < 0)
						potential[e.dst] += rcost(e);
	}

	const cost_type oo = numeric_limits<cost_type>::max();

	vector<cost_type> dist;
	vector<edge *> back;

	cost_type dijkstra(int source, int sink) {
		fill(dist.begin(), dist.end(), oo);

		typedef pair<cost_type, int> node;
		priority_queue<node, vector<node>, greater<node>> pq;

		for (pq.push({dist[source] = 0, source}); !pq.empty();) {
			node p = pq.top();
			pq.pop();

			if (dist[p.second] < p.first)
				continue;
			if (p.second == sink)
				break;

			for (edge &e : adj[p.second])
				if (e.flow < e.cap && dist[e.dst] > dist[e.src] + rcost(e)) {
					back[e.dst] = &e;
					pq.push({dist[e.dst] = dist[e.src] + rcost(e), e.dst});
				}
		}

		return dist[sink];
	}

	pair<flow_type, cost_type> max_flow(int source, int sink) {
		flow_type flow = 0;
		cost_type cost = 0;

		for (int u = 0; u < n; ++u)
			for (edge &e : adj[u])
				e.flow = 0;

		potential.assign(n, 0);
		dist.assign(n, 0);
		back.assign(n, nullptr);

		bellman_ford(source); // remove negative costs

		while (dijkstra(source, sink) < oo) {
			for (int u = 0; u < n; ++u)
				if (dist[u] < dist[sink])
					potential[u] += dist[u] - dist[sink];

			flow_type f = numeric_limits<flow_type>::max();

			for (edge *e = back[sink]; e; e = back[e->src])
				f = min(f, e->cap - e->flow);
			for (edge *e = back[sink]; e; e = back[e->src])
				e->flow += f, adj[e->dst][e->rev].flow -= f;

			flow += f;
			cost += f * (potential[sink] - potential[source]);
		}
		return {flow, cost};
	}
};


\end{lstlisting}
\end{multicols}
\begin{multicols}{2}
[\subsection{ARTICULATION POINT}\ ]
\begin{lstlisting}[language=C++]
vector<bool> visited;
vi t;
vi low;
vector<bool> art;

void dfs_art(vi adj[], int n, int p, int q)
{
    t[n] = q;
    low[n] = q++;
    visited[n] = true;

    int j = 0;

    for (int i = 0; i < adj[n].size(); i++)
    {
        if (!visited[adj[n][i]])
        {
            dfs_art(adj, adj[n][i], n, q);
            low[n] = min(low[adj[n][i]], low[n]);
            j++;

            if (low[adj[n][i]] >= t[n] && p != -1)
            {
                art[n] = true;
            }
        }
        else if (adj[n][i] != p)
        {
            low[n] = min(t[adj[n][i]], low[n]);
        }
    }

    if (p == -1)
    {
        art[n] = j >= 2;
    }
}

void articulationPoints(int V, vi adj[])
{
    visited.assign(V, false);
    t.assign(V, -1);
    low.assign(V, -1);
    art.assign(V, false);

    for (int i = 0; i < V; i++)
    {
        if (!visited[i])
        {
            dfs_art(adj, i, -1, 1);
        }
    }
}


\end{lstlisting}
\end{multicols}
\enlargethispage*{\baselineskip}

\end{document}